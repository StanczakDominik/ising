\documentclass[11pt]{article}
\usepackage{amsmath} % AMS Math Package
\usepackage{polski}
\usepackage[utf8]{inputenc}
\usepackage[procnames]{listings}
\usepackage{graphicx} % Allows for eps images
\usepackage{hyperref}
\newcommand{\png}[1]{\begin{center}\includegraphics{#1}\end{center}}
\newcommand{\largepng}[1]{\begin{center}\includegraphics[width=\linewidth]{#1}\end{center}}

\usepackage{color}
\definecolor{codegreen}{rgb}{0,0.6,0}
\definecolor{codegray}{rgb}{0.5,0.5,0.5}
\definecolor{codepurple}{rgb}{0.58,0,0.82}
\definecolor{backcolour}{rgb}{0.95,0.95,0.92}

\lstdefinestyle{mystyle}{
    backgroundcolor=\color{backcolour},
    commentstyle=\color{codegreen},
    keywordstyle=\color{magenta},
    numberstyle=\tiny\color{codegray},
    stringstyle=\color{codepurple},
    basicstyle=\footnotesize,
    breakatwhitespace=false,
    breaklines=true,
    captionpos=b,
    keepspaces=true,
    numbers=left,
    numbersep=5pt,
    showspaces=false,
    showstringspaces=false,
    showtabs=false,
    tabsize=2
}
\lstset{style=mystyle}
\title{Dwuwymiarowy model Isinga - symulacja komputerowa}
\author{Dominik Stańczak}

\begin{document}
\maketitle

\section{Model Isinga}

W roku 1924 niemiecki fizyk Ernst Ising zaproponował model, nazwany później od jego
nazwiska, mający na celu wytłumaczyć zjawiska zachodzące w ferromagnetykach, a zwłaszcza
przejście fazowe w temperaturze Curie. Model Isinga, jak powszechnie wiadomo,
opiera się na przedstawieniu spinów\footnote{Oczywiście można wykorzystać model
Isinga do modelowania innych zjawisk, lecz dla skupienia uwagi ograniczmy się
do modelowania materiałów magnetycznych.} w materiale jako dyskretnych cząstek
na siatce, obdarzonych spinem mogącym przyjmować wartości $S_i = \pm 1$.
Hamiltonian takiego układu w przypadku nieuwzględniającym zewnętrznego pola
magnetycznego przedstawia się jako

\[ H = -\sum_{i, j\neq j}{J_{ij} S_i S_j}\]

gdzie $J_{ij}$ jest tak zwaną całką wymiany, wielkością określającą siłę
wzajemnego oddziaływania między dowolnymi dwiema cząstkami, zaś sumowanie odbywa
się po wszystkich parach cząstek w układzie. Często przyjmuje się, że $J_{ij}$ ma
niezerową wartość (często $1$) wyłącznie dla najbliższych sąsiadów danej cząstki.

Należy zwrócić uwagę, że dla $J>0$ korzystna energetycznie jest sytuacja, gdy
wszystkie spiny mają identyczny kierunek - materiał jest wtedy ferromagnetyczny.
Dla $J<0$ korzystna energetycznie jest sytuacja, w której wszystkie spiny mają kierunek
przeciwny do swoich sąsiadów.

Magnetyzację układu definiuje się w prosty sposób jako sumę orientacji wszystkich
spinów w układzie:

\[M = \sum_i{S_i}\]

Ising znalazł w swojej pracy doktorskiej rozwiązanie układu w jednym wymiarze,
w którym mowa o tzw. łańcuchu Isinga. Niestety, w jednym wymiarze łańcuch Isinga
nie przejawia przejścia fazowego, zaś uporządkowanie układu, którego spodziewamy
się w systuacji ferromagnetycznej, maleje wykładniczo w czasie. Spiny są więc
zorientowane losowo, a magnetyzacja krąży wokół zera - nie jest to więc dobry model
magnesu. Ising błędnie wywnioskował, że jego model będzie się zachowywał podobnie w dowolnej liczbie wymiarów.

\section{Dwuwymiarowy model Isinga}
W roku 1944 Lars Onsager w swojej własnej pracy doktorskiej rozwiązał analitycznie
model Isinga dla dwóch wymiarów, z okresowymi warunkami brzegowymi. Jest to jednoznaczne z założeniem,
że materiał ma całkowitą symetrię translacyjną. Jak się okazało, w dwóch wymiarach model faktycznie
wykazuje przejście fazowe w temperaturze krytycznej, zakładając izotropię
(niezależność całki wymiany od kierunku oddziaływania):
\[ T_C = \frac{2}{\ln{(1+\sqrt{2})}} \]
Poniżej temperatury krytycznej układ ma stabilne minima energetyczne (stany równowagi) dla średniej
energii na cząstkę\footnote{\url{https://en.wikipedia.org/wiki/Square-lattice_Ising_model#Exact_solution}}

\[ u = <U> = -J \coth{(2\beta J)} \Big(1+\frac{2}{\pi}(2\tanh^2{(2\beta J)} -1) K(x) \Big) \]

gdzie $\beta=(k_B T)^{-1}$, $x=\sinh^2{2 \beta J}$, zaś $K(x)$ jest całką eliptyczną zupełną pierwszego rodzaju:

\[ K(x) = \int_0^{\pi/2} (1-x \sin^2(t))^{-1/2} dt\]

Należy zwrócić uwagę, że dla $T=T_C$, $K(x)=\infty$.

W stanie równowagi teoretyczna magnetyzacja na cząstkę:

\[m = <M> = [1-\sinh^{-4}{(2\beta J)}]^{1/8} \]

Układ wykazuje więc spontaniczną magnetyzację - tak, jak spodziewamy się dla
ferromagnetyka.

\section{Symulacja komputerowa}
Ze względu na prostotę swoich podstawowych reguł dwuwymiarowy model Isinga znakomicie
nadaje się do symulacji komputerowej. Siatka spinów może być w bardzo łatwy i logiczny sposób
modelowana jako dwuwymiarowa tablica liczb całkowitych o rozmiarze $NxN$. Spiny są wtedy jednoznacznie określone
przez indeksy $i, j$ z zakresu $(0,N)$ na siatce. Jako początkowy stan układu przyjmujemy losową tablicę
liczb całkowitych $S_i = \pm 1$.

Sąsiedzi danego spinu są łatwi do znalezienia jako spiny o indeksach
\[(i+1, j), (i-1,j), (i,j+1), (i,j-1)\]

W celu uwzględnienia okresowych warunków brzegowych obliczamy indeksy modulo N, na przykład
dla spinu w prawym dolnym rogu symulacji o indeksie $(N,N)$ indeksy sąsiadów to:
\[(0,N), (N-1,N), (N,0),(N,N-1)\]

\section{Dynamika układu. Algorytm Metropolis-Hastings}
Pozostaje kwestia najważniejsza - implementacja dynamiki układu. Algorytm stosowany
w tym celu jest nieskomplikowany:
\begin{enumerate}
  \item Wybieramy losowy spin jako parę indeksów $(i,j)$.
  \item Obliczamy jego energię interakcji tego spinu w przypadku, gdybyśmy zdecydowali się go przerzucić, z jego czterema sąsiadami, potencjalnie
  uwzględniając okresowe warunki brzegowe, jako sumę $E=-J S_{ij}\sum_{\text{sasiedzi}}{S_{\text{sasiad}}}$. Należy zwrócić uwagę, że różnica
  energii między tymi dwoma stanami wynosi $\Delta E = -2 E$.
  \item Na podstawie zmiany energii interakcji danego spinu decydujemy, czy należy go przerzucić.
  \item Opcjonalnie, jeśli nastąpiło przerzucenie, aktualizujemy energię oraz magnetyzację układu.
\end{enumerate}
Oczywiście niewyjaśnioną pozostaje kwestia decyzji, czy należy przerzucić spin. Istnieje jednak proste rozwiązanie
tego problemu: algorytm Metropolis-Hastings. Według tego algorytmu na podstawie reguły ``równowagi szczególnej''\footnote{Tłumaczenie własne,
nie znam bowiem polskiej literatury na ten temat.} można stwierdzić, że prawdopodobieństwo akceptacji przejścia ze stanu A (na przykład spin up)
do stanu B (spin down) powinno być proporcjonalne do $\frac{P(B)}{P(A)} = \frac{Z}{Z} e^{-\beta(E(B)-E(A))}=e^{-\beta \Delta E}$. Należy zwrócić uwagę,
że algorytm ten nie wymaga obliczania sumy statystycznej układu $Z$ - skraca się ona w trakcie dzielenia.

W praktyce wystarczy więc dla proponowanego przerzucenia spinu obliczyć czynnik boltzmannowski $P=e^{-\beta \Delta E}$, zawierający się dla dodatnich
temperatur w przedziale $(0,1)$. Następnie należy wylosować liczbę rzeczywistą z tego przedziału i porównać ją z $P$. Jeżeli wylosowana liczba jest
mniejsza od P, należy zaakceptować przejście (przerzucenie spinu). Warto zauważyć, iż dla $\Delta E < 0$ każde przejście zostaje zaakceptowane:
algorytm odwzorowuje przejście do minimum energetycznego jako stanu równowagi.
\section{Kod symulacji w Pythonie}
Program napisany jest w Pythonie 3.5 (dystrybucja Anaconda). Kod dostępny jest również
w repozytorium\footnote{\url{https://github.com/StanczakDominik/ising}}.

\lstinputlisting[language=Python]{../ising.py}

\section{Wyniki}

Wykonałem trzy oddzielne symulacje dla układu $256^2=65 536$ cząstek w temperaturach
\[0.5, T_C = \frac{2}{\ln{(1+\sqrt{2})}}, 3.5 \]
w układzie jednostek znormalizowanym do stałej Boltzmanna $k_B = 1$.
Jako przykładową demonstrację warunków początkowych, wygenerowany początkowo losowy stan:
(dla $T=1$, lecz de facto niezależnie od temperatury):

\largepng{../data/N256_T1.0/initial.png}

\subsection{$T=0.5$}

\largepng{../data/N256_T0.5/plot.png}

Dla temperatury poniżej $T_C$ następuje bardzo szybka zbieżność energii do wartości bliskiej przewidywanej teoretycznie.
Magnetyzacja również zbiega do wartości przewidywanej teoretycznie, lecz tym razem dużo, dużo wolniej. Ma to swoje uzasadnienie:
układ przez długi czas składa się nie z jednego układu o wszystkich spinach wskazujących w tę samą stronę, lecz z dyskretnych
domen magnetycznych. Te zaś są stabilne na długich skalach czasowych. Jest to zachowanie jako żywo przypominające znane z rzeczywistych magnesów.
Obrazuje to poniższe zdjęcie z animacji oraz sama animacja, pokazująca pierwsze 100 miliardów iteracji.
znadująca się pod linkiem: \url{https://youtu.be/KWMnoyFeenU}

\largepng{../data/N256_T0.5/final.png}

Czuję się również w obowiązku zademonstrować jedno z końcowych zdjęć w symulacji, demonstrujące końcowy stan równowagi:

\largepng{../data/N256_T0.5/stabilny.png}

\subsection{$T=T_C$}
\largepng{../data/N256_T2.3/plot.png}

Dla temperatury równej $T_C$, choć energia szybko zdaje się osiągać minimum,
jest to jednak minimum lokalne (na poziomie $-90k$ zamiast $-120k$, jak dla przypadku $T=0.5$). Zachodzą intensywne
oscylacje tak energii jak i magnetyzacji. Magnetyzacja zdaje się oscylować, osiągając co najwyżej połowę wartości
maksymalnej dla przypadku ferromagnetycznego.

Należy zwrócić uwagę, że algorytm Metropolis-Hastings jest znany ze
spowalniania blisko przejścia fazowego, ponieważ duża część przerzuceń spinów zostaje odrzucona. Symulacja tym
naiwnym algorytmem w tym reżimie jest więc utrudniona.

Chciałbym też zauważyć, że teoretyczna energia przewidywana równaniami Onsagera wynosi $+\infty$ i, o ile mogę to stwierzdić,
nie ma sensownego fizycznego znaczenia poza sygnalizowaniem przejścia fazowego - coś w układzie idzie ``nie tak''.

W animacji układ wygląda jak superpozycja uporzadkowanej sytuacji w magnetyku - występują w nim wyraźne domeny ferromagnetyczne - oraz
silnych szumów w paramagnetyku. Animacja pierwszych stu miliardów iteracji znajduje się pod tym linkiem: \url{https://www.youtube.com/watch?v=kHgOxCc_jPI}

\largepng{../data/N256_T2.3/final.png}

\subsection{$T=3.5$}

\largepng{../data/N256_T3.5/plot.png}

Dla temperatury powyżej temperatury krytycznej układ bardzo szybko (praktycznie natychmiastowo) zbiega do stabilnego minimum
swojej energii, przewidywanej zresztą równaniami Onsagera. Występują jednak silne oscylacje wokół tego minimum, zaś samo minimum
jest względnie wysokoenergetyczne ($-45k$ wobec $-120k$ dla przypadku ferromagnetycznego). Jedyna magnetyzacja jaka występuje w
układzie to intensywna, choć o rząd wielkości mniejsza niż w poprzednich przypadkach, oscylacja wokół zera. Układ w tej temperaturze jest
paramagnetykiem. Model Isinga działa!

Animacja wyniku tej symulacji znajduje się pod następującym linkiem: \url{https://www.youtube.com/watch?v=rCm1EI82TXk}. Wyraźnie widać, że układ zaczynający
z losowego ustawienia pozostaje losowy - oddziaływanie między spinami nie jest na tyle silne, aby utrzymać jakikolwiek porządek w układzie.

\largepng{../data/N256_T3.5/final.png}


\section{Bibliografia}
  \begin{itemize}
      \item \url{https://www.coursera.org/course/smac} - ``Statistical Mechanics -
      Algorithms and Computations'', wyśmienity internetowy kurs wprowadzający w tajniki symulacji układów statystycznych, w tym modelu Isinga.
      \item \url{https://en.wikipedia.org/wiki/Square-lattice_Ising_model} - szybki podgląd wzorów analitycznych dla modelu Isinga 2D
      \item \url{https://en.wikipedia.org/wiki/Ising_model#The_Metropolis_algorithm} - przypomnienie formalnej wersji algorytmu Metropolis-Hastings
      \item \url{https://github.com/StanczakDominik/ising} - repozytorium programu
  \end{itemize}
\end{document}
