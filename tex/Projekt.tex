\documentclass[11pt]{article}
\usepackage{amsmath} % AMS Math Package
\usepackage{polski}
\usepackage[utf8]{inputenc}
\usepackage[procnames]{listings}
\usepackage{graphicx} % Allows for eps images
\usepackage{hyperref}
\newcommand{\png}[1]{\begin{center}\includegraphics{#1}\end{center}}
\newcommand{\largepng}[1]{\begin{center}\includegraphics[width=\linewidth]{#1}\end{center}}

\usepackage{color}
\definecolor{codegreen}{rgb}{0,0.6,0}
\definecolor{codegray}{rgb}{0.5,0.5,0.5}
\definecolor{codepurple}{rgb}{0.58,0,0.82}
\definecolor{backcolour}{rgb}{0.95,0.95,0.92}

\lstdefinestyle{mystyle}{
    backgroundcolor=\color{backcolour},
    commentstyle=\color{codegreen},
    keywordstyle=\color{magenta},
    numberstyle=\tiny\color{codegray},
    stringstyle=\color{codepurple},
    basicstyle=\footnotesize,
    breakatwhitespace=false,
    breaklines=true,
    captionpos=b,
    keepspaces=true,
    numbers=left,
    numbersep=5pt,
    showspaces=false,
    showstringspaces=false,
    showtabs=false,
    tabsize=2
}
\lstset{style=mystyle}
\title{Dwuwymiarowy model Isinga - symulacja komputerowa}
\author{Dominik Stańczak}

\begin{document}
\maketitle

\section{Model Isinga}

W roku 1924 niemiecki fizyk Ernst Ising zaproponował model, nazwany później od jego
nazwiska, mający na celu wytłumaczyć zjawiska zachodzące w ferromagnetykach, a zwłaszcza
przejście fazowe w temperaturze Curie. Model Isinga, jak powszechnie wiadomo,
opiera się na przedstawieniu spinów\footnote{Oczywiście można wykorzystać model
Isinga do modelowania innych zjawisk, lecz dla skupienia uwagi ograniczmy się
do modelowania materiałów magnetycznych.} w materiale jako dyskretnych cząstek
na siatce, obdarzonych spinem mogącym przyjmować wartości $S_i = \pm 1$.
Hamiltonian takiego układu w przypadku nieuwzględniającym zewnętrznego pola
magnetycznego przedstawia się jako

\[ H = -\sum_{i, j\neq j}{J_{ij} S_i S_j}\]

gdzie $J_{ij}$ jest tak zwaną całką wymiany, wielkością określającą siłę
wzajemnego oddziaływania między dowolnymi dwiema cząstkami, zaś sumowanie odbywa
się po wszystkich parach cząstek w układzie. Często przyjmuje się, że $J_{ij}$ ma
niezerową wartość (często $1$) wyłącznie dla najbliższych sąsiadów danej cząstki.

Należy zwrócić uwagę, że dla $J>0$ korzystna energetycznie jest sytuacja, gdy
wszystkie spiny mają identyczny kierunek - materiał jest wtedy ferromagnetyczny.
Dla $J<0$ korzystna energetycznie jest sytuacja, w której wszystkie spiny mają kierunek
przeciwny do swoich sąsiadów.

Magnetyzację układu definiuje się w prosty sposób jako sumę orientacji wszystkich
spinów w układzie:

\[M = \sum_i{S_i}\]

Ising znalazł w swojej pracy doktorskiej rozwiązanie układu w jednym wymiarze,
w którym mowa o tzw. łańcuchu Isinga. Niestety, w jednym wymiarze łańcuch Isinga
nie przejawia przejścia fazowego, zaś uporządkowanie układu, którego spodziewamy
się w systuacji ferromagnetycznej, maleje wykładniczo w czasie. Spiny są więc
zorientowane losowo, a magnetyzacja krąży wokół zera - nie jest to więc dobry model
magnesu. Ising błędnie wywnioskował, że jego model będzie się zachowywał podobnie w dowolnej liczbie wymiarów.

\section{Dwuwymiarowy model Isinga}
W roku 1944 Lars Onsager w swojej własnej pracy doktorskiej rozwiązał analitycznie
model Isinga dla dwóch wymiarów, z okresowymi warunkami brzegowymi. Jest to jednoznaczne z założeniem,
że materiał ma całkowitą symetrię translacyjną. Jak się okazało, w dwóch wymiarach model faktycznie
wykazuje przejście fazowe w temperaturze krytycznej, zakładając izotropię
(niezależność całki wymiany od kierunku oddziaływania):
\[ T_C = \frac{2}{\ln{(1+\sqrt{2})}} \]
Poniżej temperatury krytycznej układ ma stabilne minima energetyczne (stany równowagi) dla średniej
energii na cząstkę\footnote{\url{https://en.wikipedia.org/wiki/Square-lattice_Ising_model#Exact_solution}}

\[ u = <U> = -J \coth{(2\beta J)} \Big(1+\frac{2}{\pi}(2\tanh^2{(2\beta J)} -1) K(x) \Big) \]

gdzie $\beta=(k_B T)^{-1}$, $x=\sinh^2{2 \beta J}$, zaś $K(x)$ jest całką eliptyczną zupełną pierwszego rodzaju:

\[ K(x) = \int_0^{\pi/2} (1-x \sin^2(t))^{-1/2} dt\]

Należy zwrócić uwagę, że dla $T=T_C$, $K(x)=\infty$.

W stanie równowagi teoretyczna magnetyzacja na cząstkę:

\[m = <M> = [1-\sinh^{-4}{(2\beta J)}]^{1/8} \]

Układ wykazuje więc spontaniczną magnetyzację - tak, jak spodziewamy się dla
ferromagnetyka.

\section{Symulacja komputerowa}
Ze względu na prostotę swoich podstawowych reguł dwuwymiarowy model Isinga znakomicie
nadaje się do symulacji komputerowej. Siatka spinów może być w bardzo łatwy i logiczny sposób
modelowana jako dwuwymiarowa tablica liczb całkowitych
\section{Metropolis-Hastings}

\section{Kod symulacji w Pythonie}
\lstinputlisting[language=Python]{../ising.py}

\section{Wyniki}

\section{Bibliografia}
  \begin{itemize}
      \item \url{https://www.coursera.org/course/smac} - MOOC ``Statistical Mechanics -
      Algorithms and Computations''
      \item \url{https://en.wikipedia.org/wiki/Square-lattice_Ising_model}
      \item \url{https://github.com/StanczakDominik/ising}
  \end{itemize}
\end{document}
